% LLNCStmpl.tex
% Template file to use for LLNCS papers prepared in LaTeX
%websites for more information: http://www.springer.com
%http://www.springer.com/lncs

\documentclass{llncs}
%Use this line instead if you want to use running heads (i.e. headers on each page):
%\documentclass[runningheads]{llncs}

\usepackage{graphicx}
\usepackage{subfig}

\begin{document}
\title{The Standard Platform League}

%If you're using runningheads you can add an abreviated title for the running head on odd pages using the following
%\titlerunning{abreviated title goes here}
%and an alternative title for the table of contents:
%\toctitle{table of contents title}

%\subtitle{Subtitle Goes Here}

%For a single author
%\author{Author Name}

%For multiple authors:
\author{ Eric Chown\inst{1}}

%If using runnningheads you can abbreviate the author name on even pages:
%\authorrunning{abbreviated author name}
%and you can change the author name in the table of contents
%\tocauthor{enhanced author name}

%For a single institute
%\institute{Institute Name \email{email address}}

% If authors are from different institutes
\institute{Bowdoin College \email{echown@bowdoin.edu}}

%to remove your email just remove '\email{email address}'
% you can also remove the thanks footnote by removing '\thanks{Thank you to...}'

\maketitle

\begin{abstract}
The Standard Platform League is unique among RoboCup soccer leagues for
its focus on software. Since all of the teams compete using the same hardware
success is theoretically predicated completely on software quality which should
make the quality judgements simpler and more objective. Growing out a league
based on the Aibo platform, the league has constantly evolved as it moves ever
closer to playing by human rules. In the past the hallmark of the league has been
a focus on individual skills such as motion, localization, and kicking at the expense
of more team oriented skills such as passing. The league has begun to address
this with the creation of the Drop-in Challenge where teams will be comprised
of robots taken from multiple teams. This new focus should force teams to work
on multi agent coordination in more abstract and general terms and promises
to create fruitful new lines of research.
\end{abstract}

\begin{keywords}
RoboCup, SPL, multi-agent cooperation, machine learning
\end{keywords}

\section{Introduction}

The Standard Platform League (SPL) was born as the ``Sony Quadruped Football League" in 1998.
The league adopted a standard platform, the Sony Aibo, as a way to make the focus
of the competition software rather than hardware. The Aibo was a relatively
inexpensive platform which afforded teams the ability to do a significant amount of scrimmaging
and testing. 

The advantages of a standard platform are many. There are prominent teams
in the league, for example, that could not produce their own robots but wish to use
robots and to work on software development. Further, the
standardization of the hardware puts teams on theoretically equal footing making the
competition ultimately about the quality of the software. Since the teams all use the
same robots differences in the speed of motion, the quality of localization, etc. can
be directly attributed to software rather than having to be disentangled from differences
in hardware.
This required focus on software
has arguably pushed software development further in the SPL than in any other humanoid
league. This can be seen, for example, by the performance of Team DARwin which has won the KidSIze
league three years in a row using a codebase developed for the SPL. That code base
has since been widely cloned in the KidSize league. The focus on software can also be
seen in the large number of research papers in the RoboCup Symposium that originate in the
SPL.

\section{History}

\subsection{The Aibo Years}

The Aibo was
an ideal platform for the original league as the robots were cheap, relatively robust,
and having four legs meant that their motion was relatively simple (compared
to two legged robots) to control. The Aibo also had a very poor CMOS camera
and was very low to the ground making vision and localization extremely
difficult. 

To combat these problems the league used brightly colored solid goals in addition
to a number of multi-colored beacons that ringed the field. While these unique
landmarks helped to make localization much simpler, ironically they made
vision something of a larger problem because they increased the number of landmarks
to be recognized as well as the number of colors that a vision system needed to
cope with. In particular Aibos had trouble with the color blue, used in both
goals and beacons, and red/orange, used in balls and uniforms. Blue
was a problem both because it was so close to so many colors that appear
around a typical AIbo game (e.g. blue jeans worn by spectators), and because
the Aibo camera had a peculiar feature called ``chromatic distortion'' that
shifted parts of the image towards the blue end of the color spectrum. Red
and orange were a problem because the balls used were close in color to the
uniforms that the robots wore. And since the robots were so low to the ground it
was easy to confuse their uniforms for balls.

The standard approach to vision was to use a color look-up table (LUT).
Teams would take images from the competition venue and use them to
``paint'' a color table. The process was slow and tedious and prone to error.

Once a color table was built most teams used a version of run-length encoding
to extract blobs of the various colors that made up the landmarks around the field.
Blobs were then examined and various sanity checks were run to see if,
for example, an orange blob was truly a ball and not part of a uniform.

Notably there were was very little use of more modern and general machine vision techniques
such as Hough transforms and the like mainly because of the slow processors
available. The circumstances fostered a situation where it made more sense
to engineer domain and hardware specific solutions rather than more general
purpose ones.

For localization nearly every team ran some variation of Monte Carlo
particle filter localization (MCL) or some kind of Kalman Filter or
multiple Kalman Filters. The advantage of the Kalman Filter approach
was again based on processing speed as it was difficult to use enough
particles to make an MCL approach work properly given the speed limitations
of the Aibo processor.

In terms of behaviors, the Four Legged League was marked by a succession
of improvements in individual skills and the development of basic behavioral
frameworks, but very little obvious teamwork was evident. On an individual
level the Aibos went from walking on their ``paws'' like normal dogs to walking
on what amounts to the forearms of their front legs. This made them much
more stable and ultimately much faster and more agile.  The dogs would
kick by putting their legs up at a 90 degree angle and then forcefully bringing
them down on the ball. This led to two important behavioral developments - the
``grab'' and the ``dodge.'' Robots would grab the ball by throwing their
chin on top of it and putting their front legs around it. This would stabilize the
ball and put it under their control. It then naturally led to a behavior where
the robot could actually move from side to side while holding the ball to
get to a better shooting angle such that attackers could actually dodge
goalies while holding the ball. The combination of these two strategies meant
that the best teams could easily score ten or more goals a game. What really
made this interesting however, was that these behaviors could be machine
learned. 

The last few years of the Four Legged League were marked by a significant
rise in machine learning \cite{chalup-07}. Teams discovered that their walks could easily
by optimized by any of a variety of machine learning techniques \cite{kohl-04} \cite{chernova-04}
Further,
individual behaviors such as the grab \cite{fidelman-07} could be learned, as well as various
aspects of color learning \cite{quinlan-04}.

\subsection{The Nao Years}

In 2008 after Sony ceased production on the Aibos, the league began
transitioning to a two legged platform and began using Aldebaran's
Nao robots. In 2008 the league ran two competitions and in 2009 it transitioned
completely to the Nao platform. Moving to the Nao solved a number of
problems that the Aibos had, but also created significant new problems.

The Naos had significantly better cameras than the Aibos and faster
processors. In addition the league was able to move to goal posts
that were true posts and eliminated the use of beacons because the
Nao's increased height enabled it to see more of the field than the
AIbos could, and notably the field lines. All of these things combined
to make vision simpler, and ultimately better localization was possible because
the Naos increased field of view made using field lines possible. Eventually
this even made it possible to color both set of goals posts yellow. While
this created interesting new localization challenges it also made vision
even simpler by eliminating another color.

The reduction in colors (e.g. no more pinks on beacons, blues on goals
or beacons) and landmarks (only one type of goal, no beacons,
simpler uniforms) as well as better cameras and faster processors has
made more experimentation in vision possible. Teams such as
Nao Team HTWK and the Nao Devils have experimented with approaches
to vision that do not use color tables at all and instead infer the colors
of the field by a combination of knowledge (e.g. most of the floor
is green) and changes in luminance.  Other teams have begun to move towards 
more modern and general machine
vision approaches such as using Hough Transforms to identify lines. Such approaches
would not have been possible on the Aibo.

What has not happened is any sort of counterbalance to make vision
harder again. There are lots of versions of this that can easily be found
in real soccer. The balls, for example, vary in color and patterns from game
to game as do team uniforms. Meanwhile, while lighting is not as carefully
controlled as it was during the Aibo years, games are still played indoors with
uniform lighting conditions.

On the other hand, the league did work hard to counterbalance the
improvements the Naos brought to localization. The biggest change
was moving to perfectly symmetric fields with regard to colors. No longer
are there any unique landmarks that will tell a robot which direction that
it is facing. Contrast that to the early years when there were two unique
goals as well as six unique beacons. Although the problem has gotten
much harder, the underlying algorithms in localization have not changed significantly
beyond the addition of being able to rely on field lines. 
Probabilistic approaches such as MCL and Kalman Filters continue
to dominate but have been refined and extended with the addition of techniques such
as shared ball models, which in turn can help robots to disambiguate
which side of the field that they are on.

By far, however, the biggest area of change and research since the
switch to the Naos is work on motion. Four legged robots do not fall
over and are relatively agile. Two legged robots can fall and thus far
are still far from agile. In the first full year with the Naos the top three
finishers at RoboCup were marked by the fact that each had developed
their own walk engines (B-Human \cite{NaoWalking-WHSR-2009}, The Northern Bites \cite{strom-10}, and the
Nao Devils \cite{czarnetzki}). Since then there have been other teams who have developed
very successful walks, e.g.  rUNSWift and Nao Team HTWK,
but the league has begun to converge on a single engine as numerous
teams now use some version of the B-Human walk engine. The reasons
for this are pretty simple: a good motion system is still the dominant factor
in robot soccer at the present. The best vision, localization, and behavior
systems in the world cannot compete against a team that is simply
faster to the ball and falls down less often.

While the motion systems used in the SPL now are vastly improved from what they were
five years ago the transition has cost the league in other areas. Robots that
are slow, not agile, and prone to falling down do not make reliable teammates 
for coordinated activities such as passing. Meanwhile the robots are
capable of kicking the ball more than the full field length. This has led
to a situation where, like in the Aibo days, there has been more emphasis
on individual skills such as kicking, than on coordination and passing.
However, recent developments in the league point to a change in this
direction. One reason for this has been the development of ``motion
kicks" where the robots can kick the ball without first coming to a stop.
Such kicks are not as powerful, but are much more useful against good teams because they
move the ball more quickly and so the ball is less likely to be stolen
by an opponent. Since they are less powerful, robots using them are not
a threat to score from literally anywhere on the field and so moving the
ball down the field in a series of kicks has become more important.
As more teams use the B-Human (or rough equivalent)
walk engine, walk speed will be cease to be the dominant differentiator
between teams and behaviors will become increasingly important. With
motion kicks increasingly this should mean positioning and passing will
start to become the factors that separate teams.

\section{The Present}

In an effort to spur development of better team play the league has
begun to experiment with ``drop-in" games. Such games consist of
teams of robots drawn randomly from the population of competing
teams. To emphasize the
importance of this competition it is required that teams compete and teams
are invited to compete even if they did not qualify for the main soccer
competition. From the perspective of the league there are two major challenges
in running such a competition. The biggest challenge is how to rank
the teams. Since one of the major goals of the competition is to increase
passing and teamwork, a scoring systems has been created with metrics
that reward such activities. Nevertheless, judging a robot's quality in
a given game is at best a difficult proposition. A poor decision may be
as much the result of bad information from teammates as from the robot's
own behavior system. The hope is that such variables will tend to even
out over the course of a number of games. It is also clear that judging
metrics will necessarily evolve over time to better meet the needs of the
league.

A second challenge is communication. Drop-in games are not even possible
unless robots use a commonly agreed upon communication
protocol so that they can share basic state information.  To facilitate these
games the league has moved to a standard message packet that each team
is required to use. The packet contains a number of standardized fields
for things such as the robot's location and the location of the ball, and it
also contains fields that can be customized for any teams own needs when
playing normal games. The construction of the packet is not trivial as
the decisions have consequences for how drop-in teams might perform.
Early versions of the packet, for example, only contained basic information
about which robot it was, where the robot was, and where the ball was. Such packets give no
sense whatsoever of a robot's intentions. Is it going to the ball? Is it playing
a particular position, etc. These things would need to be inferred by the
robot's teammates. Ultimately the league has chosen to include more information
in the packet, including where the robot is walking to, where it is shooting
to, and a field describing its intentions in terms of taking positions such
as defender or keeper. Of course the quality and even the truth of this
information has to be considered by any robots receiving it.

There are many challenges involved in making such games work and
it could open rich areas of research for the league.
For any given robot, for example, it is necessary to figure out which
of its teammates it can trust. Some will be faster than others or have
better localization systems. Some will communicate a ball location
that isn't correct, etc. One change that this system may foster is
to a more human-like position system. A theoretical advantage that robots
have over humans is that they can all run the same code and have
the same hardware. This means that players can swap positions at any
time. A single player might move from defense to striker and back to
defense again all in the space of a single point. If this is done seamlessly
it affords many advantages. In an ad hoc game, however, the amount of
coordination to pull such a system off may not be possible. 
Further, in such scenarios robots really might have different capabilities.
Therefore it makes
more sense for robots to adopt a strategy much more like humans where
players take a role initially and stick with it. That way some of the
communication that is lost by using a common and simple protocol, can
be implicitly gained back by knowing the basic roles of teammates.
A human midfielder need not communicate explicitly with all of her
teammates to know generally where they are and what they are doing
if she understands the basic principles of team soccer.

This year the league is also experimenting with a coaching robot. The
idea comes from human soccer where coaches provide general strategy
for teams from the sideline. So this year teams will be allowed to put
a robot on the sideline which can communicate with the field players. The
danger of such a system, of course, is that the coach, who will have an
excellent view of the field, could provide information that would solve
problems - e.g. where the ball is, what side of the field the ball or robots
are on, etc. To combat this, the league is experimenting with limiting
the number of communiques that the coach can provide, they style of
communication, and is also putting a time delay on its communications.
The idea is that the coach should merely be communicating overarching
strategies.
These communication limits are mirrored somewhat by a large drop in
the amount packets that field players are allowed to send. Further, all
players are required to use a new standard format for communicating
which, among other things, is necessary such that robots can work together
in the drop-in challenge.

Meanwhile solving the localization problems involved in a symmetric
field remains an area of ongoing research. To this point no best
practice has emerged, rather most teams seem to use a variety of
heuristics. The most common has probably been to rely on teammates,
but other solutions have ranged from having a team's goalie make
distinctive sounds to using a variant of the SURF algorithm to try
to build an on-the-fly map of the field's surround.

In terms of the mechanics of soccer, the league has once again progressed
back to the point where the games are five-on-five as they were at
the end of the AIbo era and the field has grown to be 9x6 meters. Typically
in recent years the league has stuck with a given size and number of
players for approximately two to three years, so it is likely that the league
will look to use larger fields with more players in the next year. While this
growth represents important movement towards the ideal of playing
on human fields it brings its own set of challenges. Some of these are
purely practical - not many labs are large enough to support a full sized
field, and of course more robots require commensurate amounts of additional
money. There is a further complication as well. In principle one of the
advantages that the SPL has over all of the rest of the RoboCup soccer
leagues is that it is indeed a standard platform. Among other things this
means that it should be possible to run regular scrimmages and even
to scrimmage other teams. This actually happened a fair amount in the later
Aibo years. For example, the Northern Bites were able to compete remotely
in the German Open in 2008. In turn, the German Team provided the
Northern Bites with runnable versions of their code in order that the Northern
Bites could scrimmage the German Team in its lab. The advantages of
such arrangements are myriad, but boil down to the fact that they way to
get better at anything is to practice. In RoboCup scrimmaging provides
essential data and opportunities for debugging. Being able to scrimmage
against other teams means that the data generated has variation. In some
ways there are opportunities for these kinds of scrimmages now more than
ever as a number of teams have taken to releasing their code on a yearly
basis. However, as the rules change the amount of work necessary to get
one of these releases running, and running under the current rules, quickly becomes prohibitive.
Further, even running a scrimmage requires ten healthy robots, a number
that very few teams can afford.

\section{The Future}

The move to two-legged robots was a case of the league needing to take a number
of steps backwards in terms of quality of play in order to make some very large
steps forward possible. The difficulty of walking stably and quickly was clearly the dominant
focus of the SPL for several years after the switch. At this point the
low hanging fruit of what the Naos can do has been reached and many of the
remaining limitations are about hardware. On the one hand such limitations
are a roadblock to further development in terms of motion, and on the other
it means that the standings of the league will not be so dependent on who
has the best walk engine, which ought to force developments in other areas.
The biggest area of need is almost certainly the further development of team play
such as passing. In order for a team to develop a successful passing strategy
there must be a combination of ability and need. On the ability side robots must
know where their teammates, as well as their opponents, are and be able to kick the ball with a fair amount
of precision. Receiving robots must be agile enough to stop and control the
ball. On the need side it must be the case that it is better for a robot to pass
the ball then to simply make a long kick at the goal or try to dribble the ball
upfield on its own.

In terms of ability many of the pieces are in place. Robots in the league localize
well enough that most robots very reliably know where their teammates are
with a relatively small amount of error. While opponent localization is a much
harder problem, many teams can now reliably identify opposing robots in their field
of view and are actively working on incorporating them into a shared model.
Kicking the ball accurately remains difficult
because of the shapes of the robot's feet as well as limitations on the movements
of their joints, differences in field conditions, etc. but robots are able to kick
the ball with some accuracy over short distances. The biggest limitation seems to
lay in the receiving robots. Even the best humans do not always make perfect passes,
but what they do well is to quickly adjust to imperfect passes. Humans can quickly
shift their feet, their motion path, whatever is necessary in order to receive and
control a pass. Robots in the SPL are simply not capable of doing much
of that yet. Passing is possible right now, it is just  not  nearly as fluid as it
is for humans. And yet there is still virtually no passing in the league.

The main reason there is so little passing seems to be that there is not enough payoff
to try it very often. A pass should provide the passing team with a clear advantage over
not passing. In RoboCup a player near the ball has three basic options: 1) shoot, 2) dribble,
and 3) pass, and right now passing is the third choice for most teams. This mainly has
to do with the fact that shooting is such a high-reward, low risk, option.
If a player is aligned on the goal there is little reason not to shoot. Players
can easily kick the ball hard enough to score from anywhere on the field. Even if the
goalie stops such a shot the ball has gone to a place where the shooting team is at an
offensive advantage, particularly since there is no offside rule and many teams
position players near the opposing goal. The only disadvantages of such kicks are
the possibilities that they can go out of bounds, and that the process of kicking is
relatively slow. Robots that take the time to kick over long distances risk having the
ball stolen by opponents. As such, a recent development of the league has been to
use motion kicks where robots are able to kick the ball without first stopping their
walk and setting their feet. These kicks are much quicker but are less powerful and
are almost akin to dribbling. Since the development of better motion engines has meant
that robots in the league are generally faster to the ball, motion kicks have become
necessary when playing the best teams. The natural evolution of these kicks is as
part of a passing system.

Passing will also become more important as the game changes. The long-term goal
of the SPL has been to be the first humanoid robot league to play 11 versus 11 games.
With more players on the field necessarily comes an increased premium on teamwork.
In the interim the league will continue to push towards games that are more realistic.
Changes expected to occur in the near term include changing the goal colors to white
and letting teams wear custom jerseys. Both will present new challenges for league vision
systems, both are necessary to the evolution of the league, but neither represent the
kind of challenge that will significantly advance the quality of play in the league. At some level
the league is still dominated by the need to solve basic problems and the best teams
are simply engineered better than other teams. The very fact that RoboCup is a competition
encourages teams to stick to ``best practices" at the expense of breaking ground on
radically new approaches. In recent years this has been reinforced as the best teams
have once again begun to regularly release their code and numerous teams have
used that code as the basis for their systems. On the one hand this is a good result
as it pushes the overall quality of play higher, allows new teams to get a working
system up and running, and forces the best teams to keep improving, on the other
hand too many teams make little or no effort to truly push the boundaries of robot
soccer. 

Going forward the league needs to figure out how to balance several things. The field
needs to get bigger and more robots need to play at a time. On the other hand, if the
trend is pushed too far too fast it will eliminate the ability of some teams to practice or
even compete. In the past the league has discussed facing the issue by forcing the
merger of teams into larger regional teams. The current move towards playing drop-in
games, however, seems to be a more pragmatic solution and one that has payoffs
in other dimensions. With the drop-in games a school no longer necessarily needs to have more
than a few robots. The downside is that practice and scrimmaging is still reduced. Long term it may be
that there is simply the need for more local competitions. Currently these competitions
are mainly seen as a tune-up for RoboCup, but perhaps with more incentives in place
they could be more useful. Such competitions could even include remote participation. Ultimately
robots are so useful as a tool of science precisely because they act in the real world. The
more data that we can get from such actions the better.

Ultimately the Standard Platform League is unique among the RoboCup soccer leagues
and it needs to work to ensure that it leverages the qualities that make it unique as much
as possible. The Drop-in challenge is a strong positive step in that direction. With the
increased emphasis on this portion of the competition teams should begin to think beyond
how to engineer the best performance for their team and more about how to create flexible
agents capable of thriving under a wider variety of circumstances and with changing
teammates. In turn this could lead the league to explore research areas that it would not have
otherwise. We are just beginning to look at different models of robot to robot communication and
it is certainly possible that that this exploration could also include human to robot communication
as well.

\section{Conclusion}

The SPL remains one of the most vibrant leagues in RoboCup. It has always been a leader in
moving humanoid soccer in increasingly realistic directions including the size of the field, the
number of robots on the field, and construction of the field. In turn the emphasis on software has
led the league to experiment with applying machine learning to a range of tasks and to develop
and refine techniques in motion, localization, vision, communication and behavior.


%the following is the current LNCS BibTex with alphabetic sorting
\bibliographystyle{splncs03} \bibliography{cog}

\end{document}
