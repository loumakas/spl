% LLNCStmpl.tex
% Template file to use for LLNCS papers prepared in LaTeX
%websites for more information: http://www.springer.com
%http://www.springer.com/lncs

\documentclass{llncs}
%Use this line instead if you want to use running heads (i.e. headers on each page):
%\documentclass[runningheads]{llncs}

\usepackage{graphicx}
\usepackage{subfig}
\hyphenation{Ro-bo-Cup Lea-gue experi-mented}
\usepackage{url}
\usepackage{amsmath}
\usepackage{hyperref}
\hypersetup{
    colorlinks,
    citecolor=black,
    filecolor=black,
    linkcolor=black,
    urlcolor=blue
}
\allowdisplaybreaks

\begin{document}
\title{The Standard Platform League}

%If you're using runningheads you can add an abreviated title for the running head on odd pages using the following
%\titlerunning{abreviated title goes here}
%and an alternative title for the table of contents:
%\toctitle{table of contents title}

%\subtitle{Subtitle Goes Here}

%For a single author
%\author{Author Name}

%For multiple authors:
\author{ Eric Chown\inst{1} and Michail G.~Lagoudakis\inst{2}}

%If using runnningheads you can abbreviate the author name on even pages:
%\authorrunning{abbreviated author name}
%and you can change the author name in the table of contents
%\tocauthor{enhanced author name}

%For a single institute
%\institute{Institute Name \email{email address}}

% If authors are from different institutes
\institute{
Bowdoin College, Brunswick, Maine, U.S.A. \ \email{echown@bowdoin.edu}
\and
Technical University of Crete, Chania, Crete, Greece \ \email{lagoudakis@ece.tuc.gr}
}

%to remove your email just remove '\email{email address}'
% you can also remove the thanks footnote by removing '\thanks{Thank you to...}'

\maketitle

\begin{abstract}
The Standard Platform League is unique among RoboCup soccer leagues for
its focus on software. Since all teams compete using the same hardware (a standard robotic platform),
success is predicated on software quality, and the shared hardware
makes quality judgments simpler and more objective. Growing out a league
based on the Sony AIBO quadruped robots, the league has constantly evolved while moving ever
closer to playing by human rules, and currently features the Aldebaran NAO humanoid robots. The hallmark of the league has been
a focus on individual agents' skills, such as perception, localization, and motion, at the expense
of more team-oriented skills, such as positioning and passing. The league has begun to address this deficiency
with the creation of the Drop-in Challenge, where robots
from multiple teams will work together. This new focus should force teams to work
on multi-agent coordination in more abstract and general terms and promises
to create fruitful new lines of research.
\end{abstract}

\begin{keywords}
RoboCup, SPL, NAO, multi-agent cooperation, machine learning
\end{keywords}

\section{Introduction}

The Standard Platform League (SPL) [\url{www.tzi.de/spl}] was referred to as the ``Sony Quadruped Robot Football League''
in the original rules published %by the league
%[\url{www.tzi.de/spl/pub/Website/History/Rules1998.pdf}] 
in 1998~\cite{rules1998} and was later renamed ``Four-Legged League'' (4LL).
The original league adopted a standard hardware platform, the Sony AIBO, as a way to make the focus
of the competition \textit{software}, rather than hardware. The AIBO was a 
robust robot platform, which afforded teams the ability to do a significant amount of development
and testing. 
AIBO was progressively replaced by the NAO robot between 2007 and 2009 and the league coined its current name. The latest edition of the league featured teams of five robots each, wearing red and blue jerseys, playing soccer autonomously on a $6\times9$ meters field with yellow goals on both sides, using an orange street-hockey ball (Figure~\ref{spl2013}).

The advantages of a standard platform are many. There are prominent teams
in the league, which could not produce their own robots, but wish to use
robots and to work on software development. Further, the
standardization of the hardware puts teams on theoretically equal footing, making the
competition ultimately about the quality of the developed software. Since the teams all use the
same robots, differences in the speed of motion, the accuracy of perception, the quality of localization, etc. can
be directly attributed to software, rather than having to be disentangled from differences
in hardware.
This required focus on software
has arguably pushed software development further in the SPL than in any other RoboCup humanoid league. This can be seen, for example, by the performance of Team DARWin which has won the KidSize
class competition of the Humanoid League three years in a row using a code base developed for the SPL. That code base
has since been widely cloned in the KidSize class. The focus on software can also be
seen in the large number of research papers presented at the RoboCup Symposium that originate in the
SPL.

\begin{figure}[t]
\centerline{
\includegraphics[width=1.0\columnwidth]{"spl2013"}
}
%\vspace{-0.1cm}
  \caption{SPL final game at RoboCup 2013 in Eidenhoven, The Netherlands.~\cite{botsport}}
  \label{spl2013}
\vspace{-0.3cm}
\end{figure} 

\section{History}

\subsection{The AIBO Years}

The AIBO was
an ideal platform for the original league, as the robots were relatively robust,
and having four legs meant that their motion was relatively simple, as compared
to two-legged robots, to control. Nevertheless, getting a small robot to play
soccer presented a number of difficult problems in the early stages, especially
with regard to vision and localization.

To help mitigate these problems the league used brightly-colored solid goals, in addition
to a number of multi-colored beacons that ringed the field. While these unique
landmarks helped to make localization much simpler, ironically they made
vision something of a larger problem, because they increased the number of landmarks
to be recognized, as well as the number of colors that a vision system needed to
discern. In particular, AIBOs had trouble with the colors blue/skyblue, used in
goals and beacons, and red/orange, used in balls and uniforms. Blue/skyblue colors
were a problem, because they were close to so many colors that appear
around a typical AIBO game (e.g.~blue jeans worn by spectators), but also because
the AIBO camera had a peculiar feature, called ``chromatic distortion'', that
shifted parts of the image towards the blue end of the color spectrum. Red
and orange colors were also a problem, because the balls used were close in color to the
uniforms that the robots wore. And, since the robots were so low to the ground, it
was easy to confuse their uniforms for balls.

The standard approach to vision was to use a color look-up table (LUT).
Teams would take images from the competition venue and use them to
``paint'' a color table. The process was slow and tedious and prone to error.
Because of this, attempts were made to improve the process with machine
learning~\cite{bruce-00,quinlan-04}
Once a color table was built, most teams used a version of run-length encoding
to extract blobs of the various colors that made up the landmarks around the field.
Blobs were then examined and various sanity checks were run to see, if,
for example, an orange blob was truly a ball and not part of a uniform.
Notably, there was very little use of modern and general machine vision techniques,
such as Hough and other transforms, mainly because of a desire to run at the
highest frame rate possible. In addition, dedicated, bright, invariant, uniform
lighting of the field was a necessity in those years for successful visual object recognition. 
%The circumstances fostered a situation where it made more sense
%to engineer domain and hardware specific solutions, rather than more general
%purpose ones.

The fixed landmarks of the field (goals, beacons, lines) provided the cues for robot localization. 
Nearly every team ran some instantiation of the generic probabilistic Markov localization approach based on the idea of Bayesian filtering for state estimation. 
To update the state (position and orientation) of each robot, some kind of Particle Filter (Monte-Carlo
Localization) or Kalman Filter (Gaussian Localization) or a combination of both was used. The advantage of Kalman Filters 
relied again on computational efficiency, as it was difficult to use enough
particles to make Particle Filters work adequately on the limited 
AIBO processor. 
Similar techniques were also used for tracking the (moving) ball on the field. 
Over the years, the localization problem became harder, by altering 
the number and the quality of landmarks; the initially six multi-colored beacons 
around the field, became four, and finally just two, while the goals were 
simplified towards colored goalposts. In the last two editions 
of the Four-Legged League (2007 and 2008), only blue and yellow were used as landmark colors. 
Lines were not utilized much as landmarks due to their inherent perceptual ambiguity 
in contrast to unique landmarks and the additional complexity required 
in properly recognizing them, given that the robots themselves were also white-colored. 

In terms of behaviors, the Four-Legged League was marked by a succession
of improvements in individual skills and the development of basic behavioral
frameworks, but very little obvious teamwork was evident. On an individual
level, the AIBOs went from walking on their ``paws'', like normal dogs, to walking
on what amounts to the forearms of their front legs. This made them much
more stable and ultimately much faster and more agile.  The dogs would
kick the ball by putting their front legs up at a 90 degree angle and then forcefully bringing
them down on the ball. This led to two important behavioral developments: the
``grab'' and the ``dodge.'' Robots would grab the ball by throwing their
chin on top of it and putting their front legs around it. This would stabilize the
ball and put it under their control. It then naturally led to a behavior, where
the robot could actually move from side to side, while holding the ball, to
get to a better shooting angle, to the effect that attackers could actually dodge
goalies and score goals by walking into the goal with the ball. The combination of these two strategies meant
that the best teams could easily score multiple goals during a game. What really
made this interesting, however, was that these behaviors could be acquired and optimized using machine
learning. 

The last few years of the Four-Legged League were marked by a significant
rise in machine learning~\cite{chalup-07}. Teams discovered that their walks could easily
be optimized by any of a variety of machine learning techniques~\cite{kohl-04,chernova-04}.
Furthermore,
individual behaviors, such as the grab~\cite{fidelman-07}, could be learned, as well as various
aspects of color learning~\cite{quinlan-04}.

\subsection{The NAO Years}

After Sony ceased production on the AIBOs in 2006, the league began
transitioning to a new robot platform. Following an open call for proposals presented at RoboCup 2007, 
the new standard platform was chosen to be the two-legged Aldebaran
NAO robot. In 2008, SPL ran two competitions (AIBO and NAO) and in 2009 it transitioned
completely to NAO. Moving to the biped NAO robot solved a number of
problems that AIBOs had, but the change also created significant new problems.

The NAOs were able to benefit from ten years of improvements in robot hardware.
For example, the NAOs has significantly better cameras than the AIBOs and faster
processors. Among other things, this enabled the league to move to goal posts
that were true posts. In addition beacons were no longer needed because the
NAO's increased height enabled it to see more of the field than the
AIBOs could.
%, most notably the field lines. 
As a result, vision became simpler and ultimately localization improved, because
the NAO's increased field of view made exploiting field lines possible. Eventually,
these developments allowed to color both sets of goals posts uniformly yellow. While
the new symmetric field created interesting new localization challenges, it also made vision
even simpler by eliminating the blue color from the landmarks.

The reduction in colors (e.g. no more pinks on beacons, blues on goals
or beacons) and landmarks (only one type of goal, no beacons,
simpler uniforms), as well as better cameras and faster processors, has
made more experimentation in vision possible. Teams, such as
NAO Team HTWK and NAO Devils, have experimented with approaches
to vision that do not use LUTs at all and instead infer the colors of the field
of the objects by a combination of knowledge (e.g. most of the floor
is green) and changes in luminance.  Other teams have begun to move towards 
modern and general machine
vision approaches, such as using Hough transforms to identify lines. Such approaches
would not have been possible on the AIBO.
What has not happened though is any sort of counterbalance to make vision
a hard problem again. Ideas in this direction can easily be found
in real soccer; the balls, for example, vary in color and patterns from game
to game, as do team uniforms. Meanwhile, even though field lighting is not as carefully
controlled as it was during the AIBO years, NAO games are still played indoors with
fairly invariant ceiling, but no dedicated, lighting.

On the other hand, the league did work hard to counterbalance the
improvements the NAOs brought to localization. The biggest change
was moving to perfectly symmetric fields with regard to colors. No longer
are there any unique landmarks that will tell a robot which direction
it is facing. The current environment represents a significant 
contrast that to the early years, when there were two uniquely-colored
goals, as well as six uniquely-colored beacons. Although the problem has gotten
much harder, the underlying algorithms in localization have not changed significantly,
besides incorporating field lines as landmarks and the need to rely on correct initialization. 
Probabilistic approaches based on Particle and Kalman Filters continue
to dominate, but have been refined and extended with the addition of techniques, such
as shared ball models and global (team) localization approaches, which in turn can help robots to disambiguate
their location between the symmetric positions on opposite sides of the field.

By far, however, the biggest area of change and research, since the
switch to the NAOs, is work on motion. Four-legged robots do not fall
over and are relatively agile. Two-legged robots can fall and thus far
are still far from being agile. In RoboCup 2009, the 2nd year of the NAO competition, the top three
finishers were marked by the fact that each had developed
their own walk engine (B-Human~\cite{NaoWalking-WHSR-2009}, Northern Bites~\cite{strom-10}, and
NAO Devils~\cite{czarnetzki}). Since then, there have been other teams who have developed
very successful walks, e.g.~rUNSWift and NAO Team HTWK,
but the league has begun to converge on a single walk engine. Numerous
teams now use some version of the B-Human engine. The reasons
for this are pretty simple: a good motion system is still the dominant factor
in robot soccer at the present. The best vision, localization, and behavior
systems in the world cannot compete against a team that is simply
faster to get to the ball and falls down less often.

While the motion systems used in the SPL now are vastly improved from what they were
five years ago, the transition has cost the league in other areas. Robots that
are slow, not agile, and prone to falling down do not make reliable teammates 
for coordinated activities, such as passing. Meanwhile, the robots are
capable of kicking the ball more than the full-field length. This has led
to a situation similar to the AIBO days, where there has been more emphasis
on individual skills, such as directed kicking, than on coordination and passing.
However, recent developments in the league point to a change towards this
direction. One reason for this has been the development of ``motion
kicks", where the robots can kick the ball without first coming to a stop.
Such kicks are not as powerful, but are much more useful against good teams, because they
move the ball quickly and so it is less likely to be stolen
by an opponent. Since motion kicks are less powerful, robots using them are not
a threat to score from literally anywhere on the field and so moving the
ball down the field in a series of kicks has become more important.
As more teams use the B-Human, or similarly reliable,
walk engine, walk speed will be cease to be the dominant differentiation factor
between teams and team behaviors will become increasingly important. With
precise kicking increasingly more prominent in the SPL, positioning of players and ball passing will
start to become the primary factors that separate teams.

\section{The Present}

In an effort to spur development of better team play, the league has
begun to experiment with ``\textit{drop-in}" games. Such games consist of
teams of robots drawn randomly from the population of competing
teams. To emphasize the
importance of this competition, it is required that all qualified teams compete and some teams
are invited to compete, even if they did not qualify for the main soccer
competition. From the perspective of the league, there are two major challenges
in running such a competition. The biggest challenge is \textit{how to rank}
the teams. Since one of the major goals of the competition is to increase interest in
passing and teamwork, a blind, peer-review scoring system has been created with metrics
that reward such teamwork activities. Nevertheless, judging a robot's quality in
a given game is, at best, a difficult proposition. A poor decision may be
the result of bad information provided by the teammates, but also a glimpse in the robot's
own behavior system. The hope is that such variables will tend to even
out over the course of a number of games. It is also clear that judging
metrics will necessarily evolve over time to better meet the needs of the
league.

A second challenge is \textit{communication}. Drop-in games are not even possible,
unless robots use a common, agreed upon communication
protocol, so that they can share basic state information.  To facilitate these
games, the league has moved to a standard message packet format that each team
is required to use in all games. The packet contains a number of standardized data fields
for information, such as the robot's location and the location of the ball, and it
also contains data fields that can be customized for any team's own needs, when
playing normal games. The construction of the packet is not trivial, as
the decisions have consequences for how well drop-in teams might perform.
Early versions of the packet, for example, only contained basic information
about the id of the robot sending the packet, where the robot was, and where the ball was. Such packets give no
sense whatsoever of a robot's intentions. Is it going to the ball? Is it playing
a particular position? And, so on. These things would need to be inferred by the
robot's teammates. Ultimately, the league has chosen to include more information
in the packet, including the robot's intended destination, where it is shooting
to (or, is trying to shoot to), and a description of its intentions in terms of taking roles/positions, such
as defender or keeper. Of course, the quality and even the truth of the
information conveyed through these messages has to be considered by any robots receiving it.

There are many challenges involved in making such drop-in games work, and drop-in games
could open rich areas of research for the league.
For any given robot, for example, it is necessary to figure out which
of its teammates it can trust. Some will be faster than others or have
better localization systems. Some will communicate a ball location
that isn't correct, etc. One change that this system may foster is
the development of a more human-like position system. A theoretical advantage that robots
have over humans is that they can all run the same code and have
the same hardware, meaning that players can swap positions at any
time. A single player might move from defense to striker and back to
defense again, all in the space of a single point. If this is done seamlessly,
it affords many advantages. In an ad hoc game, however, the amount of
coordination to pull such a system off may not be achievable. 
Furthermore, in such scenarios robots might really have different capabilities.
Therefore, it makes
more sense for robots to adopt a human-like strategy, where
players take a role in the beginning and stick with it. That way some of the
communication that is lost by using a simple common protocol, can
be implicitly gained back by knowing the basic roles of teammates.
A human midfielder need not communicate explicitly with all of her
teammates to know generally where they are and what they are doing,
if she understands the basic principles of team soccer.

This year the league is also experimenting with a \textit{coaching robot}. The
idea comes from human soccer, where coaches provide general strategy
for teams from the sideline. So, this year teams will be allowed to place
a coaching robot on the sideline, where it can observe the game and communicate with the field players. The
danger of such an addition, of course, is that the coach, who will have an
fairly good view of the field, could provide solve
problems for the individual players, such as which side of the field
that they are on. To combat this, the league is experimenting with limiting
the number and the style of communiques that the coach can provide, 
as well as putting a time delay on coach-player communications.
The idea is that the coach should merely be communicating overarching
strategies, not real-time state information. 
These communication limits are mirrored somewhat by a large drop in
the amount of packets that field players are allowed to send
and the requirement to use the new standard message format for communicating.

Meanwhile, solving the localization problems involved in a symmetric 
and fairly populated 
field remains an area of ongoing research. To this point no best
practice has yet emerged, rather most teams seem to use a variety of
heuristics. The most common has probably been to rely on teammates. For
example a goalie can inform teammates when the ball is on its side.
Other solutions have ranged from having a team's goalie make
distinctive sounds to reveal the location of the home goal to using a variant of the SURF algorithm to try
to build an on-the-fly map of the field's surroundings.

In terms of the mechanics of soccer, the league has once again progressed
back to the point where the games were at
the end of the AIBO era; there are five players on each side and the field has grown to be $9\times 6$ meters. Typically,
the league sticks with a given field size and number of
players for approximately two to three years, so it is likely that the league
will look to use larger fields with more players in the next year. While this
growth represents important progress towards the ideal of playing
on human fields, it brings its own set of challenges. Some of these are
purely practical---not many labs are large enough to support a full-sized
field and, of course, more robots require commensurate amounts of additional
money. There is a further complication as well having to do with how
much teams can run practice games. 

In principle, one of the
advantages SPL has over all other RoboCup soccer
leagues is the common standard robot platform. Among other things, this
means that it should be possible to run regular practice games between players of the same team and even
to play practice games against other teams, if their code (or their executables) are available. This actually happened a fair amount of times in the last few
AIBO years. For example, the Northern Bites were able to compete remotely
in the RoboCup German Open competition in 2008. In turn, the German Team provided the
Northern Bites with executable versions of their code so that the Northern
Bites could play practice games against the German Team in their lab. The advantages of
such arrangements are myriad, but boil down to the fact that the way to
get better at anything is to practice. In RoboCup, practice games provide
essential data and opportunities for debugging, considering the variation offered by playing
against other teams. In some
ways there are more opportunities for these kinds of practice games now than
ever, as several teams release their code on a yearly
basis. However, as the league rules, including field dimensions and number of players, change, the amount of work necessary to get
one of these releases running, obeying the updated rules, quickly becomes prohibitive.
Further, even with the current settings, running a full practice game requires a total of 10 healthy robots, a number
that very few teams can afford.

\section{The Future}

The move to two-legged robots was a case of the league needing to take a number
of steps backwards in terms of quality of play in order to make some very large
steps forward possible. The difficulty of walking stably and quickly was clearly the dominant
focus of the SPL for several years after the switch. At this point, the
low-hanging fruit of what the NAOs can do has more or less been reached and many of the
remaining limitations can now be attributed to the hardware. On the one hand, such limitations
are a roadblock to further development in terms of motion, but, on the other,
it means that the standings of the league will not be so dependent on who
has the best walk engine, which ought to force developments in other areas.
The biggest area with the most potential for future developement is almost certainly team play,
such as team formations, role assignment, player positioning, and ball passing. In order for a team to develop a successful passing strategy,
there must be a combination of ability and need. On the ability side, robots must
know where their teammates, as well as their opponents, are and be able to kick the ball with a fair amount
of precision. Receiving robots must be agile enough to stop and control the
ball. On the need side, it must be the case that it is better for a robot to pass
the ball than to simply make a long kick towards the opponent goal or try to dribble the ball
upfield on its own.

In terms of ability, many of the pieces are in place. Robots in the league localize
well enough that most robots very reliably know where their teammates are
with a relatively small amount of error. While opponent localization is a much
harder problem, many teams can now reliably identify opposing robots in their field
of view and are actively working on incorporating them into a shared model.
Kicking the ball accurately remains difficult,
because of the shapes of the robots' feet, limitations on the movements of their joints, and differences in field conditions. Nevertheless, robots are able to kick
the ball with some accuracy over short distances. The biggest limitation seems to
lay in the receiving robots. Even the best humans do not always make perfect passes,
but human receivers do quickly adjust to imperfect passes. Humans can quickly
shift their feet, their motion path, or whatever is necessary in order to receive and
control a pass. Robots in the SPL are simply not capable of doing much
of that yet. Simple passing, not  nearly as fluid as it
is for humans, still remains to be seen.% in the SPL.

In terms of need, a major reason there is so little passing seems to be that there is not enough payoff
to try it very often. A pass should provide the passing team with a clear advantage over
not passing. In RoboCup, a player near the ball has three basic options: shoot, dribble,
or pass; right now, passing is the third choice for most teams. This mainly has
to do with the fact that shooting is such a high-reward, low-risk option.
If a player is aligned toward the goal, there is little reason not to shoot. Players
can easily kick the ball hard enough to score from anywhere on the field. Even if the
goalie stops such a shot, the ball has gone to a place where the shooting team is at an
offensive advantage, particularly since there is no offside rule and many teams
position players near the opposing goal. The only disadvantage of such long shots is
the possibility that the ball goes out of bounds; a rule introduced in recent years to discourage random shooting dictates that, in such a case, the ball is replaced in the field one meter behind the kicking robot. On top of that, robots that take the time to stop and set their feet to kick over long distances risk having the
ball stolen by opponents. Given the lack of precise kicking, the next best option is to dribble, possibly using motion kicks, which are less powerful, but much quicker and almost akin to dribbling. 
Some addition to the rules may be required to discourage continuous dribbling and encourage work on the next option (passing). However, this may naturally occur, as the number of players and the field dimensions change. 
%Since the development of better motion engines has meant
%that robots in the league are generally faster to the ball, motion kicks have become
%necessary when playing the best teams. The natural evolution of these kicks is as
%part of a passing system.

One long-term goal
of the SPL has been to be the first humanoid robot league to play 11 versus 11 players' games.
With more players on the field necessarily comes an increased premium on teamwork.
In the interim the league will continue to push towards games that are more realistic.
Changes expected to occur in the near term include changing the goal colors to white
and letting teams wear custom jerseys. Both will present new challenges for league vision
systems, and both are necessary to the evolution of the league, but neither represents the
kind of challenge that will significantly advance the quality of play in the league. 
%At some level
%the league is still dominated by the need to solve basic problems and the best teams
%are simply engineered better than other teams. 

The very fact that RoboCup is a competition
encourages teams to stick to ``best practices" at the expense of breaking ground on
radically new approaches. In recent years this has been reinforced, as the best teams
have once again begun to regularly release their code and numerous other teams have
used that code as the basis for their systems. On the one hand, this is desirable,
as it pushes the overall quality of play higher, allows new teams to get a working
system up and running, and forces the best teams to keep improving. On the other
hand, %too 
many teams make little or no effort to truly push the boundaries of robot
soccer. 

Going forward the league requires balancing several things. The field
needs to get bigger and more robots need to play at a time. On the other hand, if the
trend is pushed too far/fast it will eliminate the ability of some teams to practice or
even compete. In the past the league has discussed facing the issue by forcing the
merger of teams into larger regional teams. The current move towards playing drop-in
games, however, seems to be a more pragmatic solution and one that has payoffs
in other dimensions. With drop-in games established, each research team no longer necessarily needs to have more
than a few robots. The downside is that practice games are practically eliminated. In the long term, it may raise 
the need for more local competitions. Currently these local competitions
are mainly seen as a tune-up for the main RoboCup event, but perhaps with more incentives in place
they could be more useful and could even include remote participation. Ultimately,
robots are so useful as a tool of science, precisely because they act in the real world. The
more data that we can get from such actions the better.

The Standard Platform League 
needs to work to ensure that it leverages the qualities that make it unique among RoboCup leagues as much
as possible. The Drop-in challenge is a strong, positive step in that direction. With the
increased emphasis on this portion of the competition teams should begin to think beyond
how to engineer the best performance for their team and more about how to create flexible
agents capable of thriving under a wider variety of circumstances and with changing
teammates. In turn, this could lead the league to explore research areas that it would not have considered
otherwise. 

To provide a specific example of how such research exploration can be realized, we note that 
any kind of communication between players in the SPL currently relies solely on the 
availability of a reliable wireless network. This is, however, an artificial communication 
channel, missing in real soccer games, whereby human players exchange and share information 
using audiovisual signals they generate. 
In addition, experience from the most recent RoboCup competitions indicates that even the 
current state-of-the-art technology in wireless communication is insufficient to support 
reliably the needs of SPL games. In the recent 2012 and 2013 SPL games many robots could not 
even connect to the network, players could not receive game state from the game controller, 
data packets were constantly missing, teammates could not communicate to coordinate, and 
eventually most efforts for successful coordination among team players were gone. Since 
communication is obviously necessary to advance teamwork, the key question that the league 
may need to address in the future is the following: How could robots exchange and share 
information, if there was no wireless or any other kind of network in the SPL field? Could 
they use the physical world? Could they generate visual and/or auditory cues that could be 
perceived by other robots and thus convey information in a more natural and human-like way? 

As a concrete example towards audiovisual communication, consider the problem of generating 
an intuitive visual signal for sharing information about the most important element in a 
soccer game: the current position of the ball in the field
%\footnote{Part of the described 
%scenario was demonstrated in the SPL Open Challenge competition at RoboCup 2012 by SPL team Kouretes.}. 
A robot having direct visual observation of the ball can assist its teammates by 
generating a visual signal that reveals the current position of the ball. The most natural 
and intuitive visual signal one could imagine is to fully extend one of the two arms, so that 
it points directly to the ball. Another robot could visually perceive this signal through its 
camera and use the vector defined by the first robot's arm to pinpoint the position of the 
ball. Note that this kind of information sharing about the position of the ball does not 
necessarily require any kind of localization of the two robots in the field or relative to 
each other. The perceiving robot could simply ``follow'' visually the line indicated by the 
pointing robot to locate the ball. Furthermore, such a visual signal could be used for 
directing attention to any desired position or object in the field; a lead player could 
direct a teammate towards a certain position in the field and a future robot referee could 
easily point out the player that was penalized or the place where the ball went out of 
bounds! Considering also the new addition of a robot coach on the sideline, images of real 
human coaches would come to mind, if one watched the robot coach pointing to positions and 
players, possibly ``shouting'' instructions at the same time. 

The elimination of the (artificial) wireless communication network will open up opportunities 
for researching several human-like behaviors, which may not be strictly restricted to soccer. 
Teams will have to develop their own (or adopt existing) sign languages to convey information 
between their players, possibly taking into account the fact that opponents are also 
``watching'' and information may have to be encoded. Robot perception will not be limited 
anymore on just correct object recognition (ball, goals, lines, players, etc.), but will also 
have to succeed in correctly recognizing postures, signals, gestures, sounds, even complete 
``oral'' sentences in a human or even in an (artificial) robot language. Needless to say that 
all competing players will have to obey referee instructions, signaled with the sounds of a 
whistle, possibly complimented with details provided ''orally''. Under the described setting, 
not only will human spectators be naturally engaged into the game play strategies of each 
team, but also the technology developed could be further exploited in numerous human-robot 
and robot-robot interaction situations beyond robot soccer. 


%We are just beginning to look at different models of robot-to-robot communication and
%it is certainly possible that that this exploration could also include human-to-robot communication as well.

\begin{figure}[t]
\centerline{
\includegraphics[width=\columnwidth]{"spl2013people"}
}
%\vspace{-0.1cm}
  \caption{SPL participants at RoboCup 2013 (picture credit: Brad Hall).}
  \label{spl2013people}
\vspace{-0.3cm}
\end{figure} 

\section{Conclusion}

The SPL remains one of the most vibrant leagues in RoboCup (Figure~\ref{spl2013people}). It has always been a leader in
advancing humanoid robot soccer in increasingly realistic directions in terms of the field dimensions, the
number of robots on the field, and the setup of the field. In turn, the emphasis on software has
led the league to develop
and refine techniques in motion, vision, localization, communication, behavior, and coordination, but also to experiment with applying machine learning and optimization techniques to a range of tasks. In the present the league is
continuing to lead as it experiments with the drop-in challenge and with coaching robots. In the future,
it will continue to push the boundaries of what robots can do on a soccer field.


%the following is the current LNCS BibTex with alphabetic sorting
\bibliographystyle{splncs03} \bibliography{cog}

\end{document}
